\documentclass[11pt]{article}

\usepackage{extras} % Se extras.sty

\begin{document}
\begin{titlepage}
\begin{center}

{\Large\bfseries TSEA56 - Kandidatprojekt i elektronik \\ LIPS Efterstudie}

\vspace{5em}

Version 1.0

\vspace{5em}
Grupp 4 \\
\begin{tabular}{rl}
Hynén Ulfsjöö, Olle&\verb+ollul666+
\\
Wasteson, Emil&\verb+emiwa068+
\\
Tronje, Elena&\verb+eletr654+
\\
Gustafsson, Lovisa&\verb+lovgu777+
\\
Inge, Zimon&\verb+zimin415+
\\
Strömberg, Isak&\verb+isast763+
\\
\end{tabular}

\vspace{5em}
\today

\vspace{16em}
Status
\begin{longtable}{|l|l|l|} \hline

Granskad & - & - \\ \hline
Godkänd & - & - \\ \hline
 
\end{longtable}

\end{center}
\end{titlepage}

\pagebreak
\begin{center}

\section*{PROJEKTIDENTITET}
2016/VT, Undsättningsrobot Gr. 4

Linköpings tekniska högskola, ISY
\vspace{5em}

\begin{tabular}{|l|l|l|l|} \hline
\textbf{Namn} & \textbf{Ansvar} & \textbf{Telefon} & \textbf{E-post}  \\ \hline 
Isak Strömberg (IS) & Projektledare & 073-980 38 50 & isast763@student.liu.se \\ \hline
Olle Hynén Ulfsjöö (OHU)& Dokumentansvarig & 070-072 91 84 & ollul666@student.liu.se \\ \hline
Emil Wasteson (EW) & Hårdvaruansvarig & 076-836 61 66 & emiwa068@student.liu.se \\ \hline
Elena Tronje (ET) & Mjukvaruansvarig & 072-276 92 93 & eletr654@student.liu.se \\ \hline
Zimon Inge (ZI)& Testansvarig & 070-171 35 18 & zimin415@student.liu.se \\ \hline
Lovisa Gustafsson (LG) & Leveransansvarig & 070-210 32 53 & lovgu777@student.liu.se \\ \hline
\end{tabular}


E-postlista för hela gruppen: isast763@student.liu.se

\vspace{5em}
Kund: ISY, Linköpings universitet 
tel: 013-28 10 00, fax: 013-13 92 82 \\
Kontaktperson hos kund: Mattias Krysander \\
tel: 013-28 21 98, e-post: matkr@isy.liu.se \\

\vspace{5em}
Kursansvarig:  Tomas Svensson\\
tel: 013-28 13 68, e-post: tomass@isy.liu.se \\
Handledare: Peter Johansson \\
tel: 013-28 13 45, e-post: peter.a.johansson@liu.se
\end{center}
\pagebreak

\tableofcontents

\pagebreak

\section*{Dokumenthistorik}
\begin{table}[h]
\begin{tabular}{|l|l|l|l|l|} \hline

\textbf{Version} & \textbf{Datum} & \textbf{Utförda förändringar} & \textbf{Utförda av} & \textbf{Granskad} \\ \hline
1.0 & - & Första utkastet & Grupp 4 & - \\ \hline
\end{tabular}
\end{table}

\pagebreak
\pagenumbering{arabic}

\section{Tidsåtgång}
Vid projektets start tilldelades gruppen en buffert på 230 timmar per person. Det var av stor vikt att denna tid inte fick överskridas och att fördelningen av arbete var jämn. Nedan följer en utvärdering över projektets tidsåtgång.

\subsection{Arbetsfördelning}
Generellt sett har alla gruppmedlemmar tagit ansvar och drivit arbetet framåt. Vid projektets start hade medlemmarna dock olika förkunskaper, vilket ledde till att de som hade mest kunskap tog på sig att genomföra mer arbete. Under projektets gång jämnade det ut sig mer och mer och medlemmarna fördjupade sig inom olika områden. Det blev också så att några lade mer tid under vissa veckor medan andra mer under andra veckor, vilket är naturligt eftersom olika områden kräver olika mycket tid vid olika tillfällen.

\subsection{Tidsåtgång jämfört med planerad tid}
Det var svårt att innan projektet hade startat ta ställning till om 230 timmar per person skulle vara tillräckligt med tid. Det visade sig också att vissa aktiviteter tog mycker mer tid än de var planerade för medan andra aktiviteter gick betydligt snabbare att genomföra. Totalt sett stämde den planerade tidsåtgången ganska väl med den faktiska tidsåtgången, se figur \ref{}.

\section{Analys av arbete och problem}
Under projektets gång uppkom en del problem, både när det gäller tekniken och kommunikationen. Detta avsnitt beskriver hur detta hanterades och förebyggdes. 

\subsection{Vad hände under de olika faserna?}
Den förberedande fasen var omfattande och lade en mycket bra grund inför det fortsatta arbetet. Gruppen lade mycket energi på att göra en bra systemskiss och designspecifikation vilket sedan underlättade det praktiska arbetet. Alla i gruppen ser det som väl investerad tid.

Det praktiska arbetet innefattade konstruktion av roboten och programutveckling av mjukvaran. En stor del av denna fas gick åt till att felsöka och att testa robotens olika funktioner. 

Efterfasen bestod precis som den förberedande fasen till stor del av dokumentation. Därutöver genomfördes en presentation av roboten och opponering på en annan projektgrupps arbete.

\subsection{Hur vi arbetade tillsammans?}
Överlag fungerade samarbetet inom gruppen väldigt bra. Alla var noggranna med kommunikationen vilket underlättade för övriga i gruppen. Projektledaren tog det ansvar som förväntades och var det några oklarheter tog gruppen gemensamma beslut. Det öppna klimatet bidrog till en bra gruppdynamik och gjorde att det aldrig uppstod några större konflikter.

\subsection{Hur använde vi projektmodellen?} 
Projektet genomfördes enligt LIPS-modellen. Den låg som grund till alla dokument som skrevs och var bra att ha som utgångspunkt i projektets moment. 

\subsection{Hur fungerade relationen med beställaren}
Relationen med beställaren var god under hela projektet. Under mötena under projektets gång gavs konstruktiv kritik och beställaren hade hela tiden en positiv inställning till nya idéer. 

\subsection{Hur fungerade relationen med handledaren}
Kontakten med handledaren fungerade mycket bra. Förutom att handledaren tillhandahöll material var han även  tillmötesgående och var alltid öppen för att diskutera problem eller alternativa lösningar under hela projektet. På grund av vissa omständigheter var handledaren inte alltid tillgänglig vissa perioder men då togs hjälp av andra handledare. Önskvärt hade kanske varit att endast ha kontakt med en handledare men det löste sig väldigt bra ändå.

\subsection{Tekniska framgångar och problem}
Ett problem som uppstod på var när alla moduler skulle integreras med varandra. De fungerade var för sig men vid integreringen störde modulerna varandra vilket gjorde att roboten inte betedde sig som önskat. Ett annat problem var att golvet i tävlingslokalen hade betydligt mindre friktion än de golv som roboten till en början testkördes på. 

Några tekniska framgångar som är värda att nämnas är:
\begin{itemize}
\item Fästena av IR-sensorerna som skrevs ut med hjälp av 3D-skrivare.
\item Regleringen implementerades bra vilket gjorde att roboten var snabb och aldrig behövde stanna upp för att tänka. Det var därmed mycket tillfredställande att se roboten köra. 
\item Avsökningsalgoritmen fungerade direkt när den implementerades, vilet fås se som en bedrift med tanke på hur omfattande algoritmen är. 
\end{itemize} 

\section{Måluppfyllelse}
I följande avsnitt beskrivs det huruvida projektets mål har uppnåtts och hur omgivningen har påverkat projektets utfalll.

\subsection{Vad har uppnåtts?}
Alla krav av prioritet 1 har uppnåtts. Några av kraven av prioritet två har det börjat att arbetas med men inget av dem har blivit uppfyllt.

\subsection{Hur fungerade leveransen?}
Skrivs efter tävling?

\subsection{Hur har studiesituationen påverkat projektet?}
Samtliga i gruppen har läst en kurs parallellt med projektet och några har även haft andra engagemang på skolan. Projektet har dock inte lidit speciellt mycket av detta eftersom projektet oftast har prioriterats högre än övriga aktiviteter. Däremot har projektet påverkat resultatet i den parallella kursen som alla i gruppen hade behövt lägga mer tid på.  

\section{Sammanfattning}
Överlag är alla i gruppen mycket nöjda med projektets arbetsgång och resultat. Det har varit en rolig och lärorik kurs som har ställt krav på teknisk kunskap så väl som att arbeta i ett större projekt. Nedan sammanfattas de enligt gruppen viktigaste erfarenheterna och råd till de som ska genomföra ett liknande projekt i framtiden.

\subsection{De tre viktigaste erfarenheterna}
\begin{itemize}
\item Vikten av kommunikation och samarbete
\item Bra förarbete/framförhållning
\item Hur det är att arbete i ett stort projekt
\end{itemize} 

\subsection{Goda råd till de som ska utföra ett liknande projekt}
För att projektet ska ha de bästa förutsättningarna för att lyckas, och för att arbetsbelastningen inte ska bli för stor under genomförandefasen, är det viktigt att lägga ner tid och energi på att göra en bra systemskiss och designspecifiaktion. Det kommer framförallt underlätta när det ska börja viras och skrivas kod. 

Det är även viktigt att ha en tydlig tidplan att utgå ifrån, annars riskerar det att gå åt onödig tid åt att varje dag/vecka bestämma vad som ska göras härnäst.

Kommentera koden enhetligt och utförligt. Det kommer skrivas många rader och det är omöjligt att komma ihåg alla variabler och metoder. Det blir även lättare för andra att sätta sig in i koden om den är bra kommenterad. 

Ha tydliga riktlinjer för gruppen. Vad är okej och vad är inte okej? Vilka tider förväntas en att vara på plats i skolan? Vem har ansvar för de olika aktiviteterna? Att vara på det klara med vad som gäller gör att många konflikter kan undvikas och arbetsklimatet blir bättre.

Avsätt en buffert för felsökning och integration av modulerna. Problem kommer att uppkomma och att hitta var felet ligger kräver ofta mer tid än förväntat. Många av dessa fel uppstår när modulerna ska integreras med varandra. Operationer kan fungera felfritt var för sig men krångla när allt ska ske samtidigt. Att vara förberedd på det gör att det går att undvika stress i slutet av projektet. 

\end{document}