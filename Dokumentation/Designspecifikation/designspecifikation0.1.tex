\documentclass[11pt]{article}

\usepackage{extras} % Se extras.sty

\begin{document}
\begin{titlepage}
\begin{center}

{\Large\bfseries TSEA56 - Kandidatprojekt i elektronik \\ LIPS Designspecifikation}

\vspace{5em}

Version 0.1

\vspace{5em}
Grupp 4 \\
\begin{tabular}{rl}
Hynén Ulfsjöö, Olle&\verb+ollul666+
\\
Wasteson, Emil&\verb+emiwa068+
\\
Tronje, Elena&\verb+eletr654+
\\
Gustafsson, Lovisa&\verb+lovgu777+
\\
Inge, Zimon&\verb+zimin415+
\\
Strömberg, Isak&\verb+isast763+
\\
\end{tabular}

\vspace{5em}
\today

\vspace{16em}
Status
\begin{longtable}{|l|l|l|} \hline

Granskad & - & - \\ \hline
Godkänd & - & - \\ \hline
 
\end{longtable}

\end{center}
\end{titlepage}

\pagebreak
\begin{center}

\section*{PROJEKTIDENTITET}
2016/VT, Undsättningsrobot Gr. 4

Linköpings tekniska högskola, ISY
\vspace{5em}
\begin{center}

\begin{tabular}{|l|l|l|l|} \hline
\textbf{Namn} & \textbf{Ansvar} & \textbf{Telefon} & \textbf{E-post}  \\ \hline 
Isak Strömberg (IS) & Projektledare & 073-980 38 50 & isast763@student.liu.se \\ \hline
Olle Hynén Ulfsjöö (OHU)& Dokumentansvarig & 070-072 91 84 & ollul666@student.liu.se \\ \hline
Emil Wasteson (EW) & Hårdvaruansvarig & 076-836 61 66 & emiwa068@student.liu.se \\ \hline
Elena Tronje (ET) & Mjukvaruansvarig & 072-276 92 93 & eletr654@student.liu.se \\ \hline
Zimon Inge (ZI)& Testansvarig & 070-171 35 18 & zimin415@student.liu.se \\ \hline
Lovisa Gustafsson (LG) & Leveransansvarig & 070-210 32 53 & lovgu777@student.liu.se \\ \hline
\end{tabular}

\end{center}

E-postlista för hela gruppen: isast763@student.liu.se

\vspace{5em}
Kund: ISY, Linköpings universitet \\
tel: 013-28 10 00, fax: 013-13 92 82 \\
Kontaktperson hos kund: Mattias Krysander \\
tel: 013-28 21 98, e-post: matkr@isy.liu.se \\

\vspace{5em}
Kursansvarig:  Tomas Svensson\\
tel: 013-28 13 68, e-post: tomass@isy.liu.se \\
Handledare: Peter Johansson \\
tel: 013-28 13 45, e-post: peter.a.johansson@liu.se
\end{center}
\pagebreak

\tableofcontents

\pagebreak

\section*{Dokumenthistorik}
\begin{table}[h]
\begin{tabular}{|l|l|l|l|l|} \hline

\textbf{Version} & \textbf{Datum} & \textbf{Utförda förändringar} & \textbf{Utförda av} & \textbf{Granskad} \\ \hline
0.1 & - &  Första utkastet & Grupp 4 & - \\ \hline
\end{tabular}
\end{table}

\pagebreak
\pagenumbering{arabic}

\begin{flushleft}
\section{Inledning}
\lipsum

\subsection{Det totala systemet}
\lipsum

\subsection{Sensorer}
\lipsum

\subsection{Ställdon}
\lipsum

\pagebreak
\section{Delmodul 1 - Huvudmodul}
\lipsum

\subsection{Detaljerad beskrivning}
\lipsum

\subsection{Hårdvara}
\lipsum

\subsection{Mjukvara}
\lipsum

\pagebreak
\section{Delmodul 2 - Sensormodul}
Sensormodulens uppdrag är att kommunicera med både sensorerna och huvudmodulen. Mätdata samplas med konstant frekvens och konverteras därefter till motsvarande SI-enhet. För avståndssensorerna innebär detta att den analoga inspänningen konverteras till ett digitalt värde i enheten meter.

Den främre avståndssensorn (laser-sensorn) placeras på ett roterande servo så att den kan \emph{scanna} korridoren roboten befinner sig i. Detta förutsätter att servot går att styra så att vinkelutslaget går att beräkna med godtycklig precision. Tillåter inte servot en sådan styrning placeras laser-sensorn så att den konstant pekar rakt framåt.

Vinkelhastighets-sensorerna används för att beräkna vinkelutslag då roboten tar en sväng. Mätdatan behöver därför integreras under ett tidsintervall och konverteras till en vinkel. Hur den nödställde ska representeras är ännu inte fastställt. Förslagsvis används en IR-fyr i kombination med en IR-sensor. Sensormodulens uppdrag blir då att ta beslutet om när den nödställde är funnen.

\subsection{Detaljerad beskrivning}
\lipsum

\subsection{Hårdvara}
\lipsum

\subsection{Mjukvara}
\lipsum

\pagebreak
\section{Delmodul 3 - Styrmodul}
\lipsum

\subsection{Detaljerad beskrivning}
\lipsum

\subsection{Hårdvara}
\lipsum

\subsection{Mjukvara}
\lipsum

\pagebreak
\section{Delmodul 4 - Datormodul}
\lipsum

\subsection{Detaljerad beskrivning}
\lipsum

\subsection{Hårdvara}
\lipsum

\subsection{Mjukvara}
\lipsum

\pagebreak
\section{Intermodulär kommunikation}
\lipsum

\subsection{I\textsuperscript{2}C-buss}
\lipsum

\subsection{Bluetooth\textsuperscript{\circledR}-kommunikation}
\lipsum

\pagebreak
\section{Implementering}
\lipsum

\end{flushleft}

\end{document}