\documentclass[11pt]{article}

\usepackage{extras} % Se extras.sty

\begin{document} 
\begin{titlepage}
\begin{center}

{\Large\bfseries TSEA56 - Kandidatprojekt i elektronik \\ LIPS Förstudie: Kommunikation}

\vspace{5em}

Version 0.1

\vspace{5em}
Grupp 4 \\
\begin{tabular}{rl}
Tronje, Elena&\verb+eletr654+
\\
Gustafsson, Lovisa&\verb+lovgu777+
\\
\end{tabular}

\vspace{5em}
\today

\vspace{16em}
Status
\begin{longtable}{|l|l|l|} \hline

Granskad & - & - \\ \hline
Godkänd & - & - \\ \hline
 
\end{longtable}

\end{center}
\end{titlepage}

\pagebreak
\begin{center}

\section*{PROJEKTIDENTITET}
2016/VT, Undsättningsrobot Gr. 4

Linköpings tekniska högskola, ISY
\vspace{5em}
\begin{center}

\begin{tabular}{|l|l|l|l|} \hline
\textbf{Namn} & \textbf{Ansvar} & \textbf{Telefon} & \textbf{E-post}  \\ \hline 
Isak Strömberg (IS) & Projektledare & 073-980 38 50 & isast763@student.liu.se \\ \hline
Olle Hynén Ulfsjöö (OHU)& Dokumentansvarig & 070-072 91 84 & ollul666@student.liu.se \\ \hline
Emil Wasteson (EW) & Hårdvaruansvarig & 076-836 61 66 & emiwa068@student.liu.se \\ \hline
Elena Tronje (ET) & Mjukvaruansvarig & 072-276 92 93 & eletr654@student.liu.se \\ \hline
Zimon Inge (ZI)& Testansvarig & 070-171 35 18 & zimin415@student.liu.se \\ \hline
Lovisa Gustafsson (LG) & Leveransansvarig & 070-210 32 53 & lovgu777@student.liu.se \\ \hline
\end{tabular}

\end{center}

E-postlista för hela gruppen: isast763@student.liu.se

\vspace{5em}
Kund: ISY, Linköpings universitet \\
tel: 013-28 10 00, fax: 013-13 92 82 \\
Kontaktperson hos kund: Mattias Krysander \\
tel: 013-28 21 98, e-post: matkr@isy.liu.se \\

\vspace{5em}
Kursansvarig:  Tomas Svensson\\
tel: 013-28 13 68, e-post: tomass@isy.liu.se \\
Handledare: Peter Johansson \\
tel: 013-28 13 45, e-post: peter.a.johansson@liu.se
\end{center}
\pagebreak

\tableofcontents

\pagebreak
\section*{Dokumenthistorik}
\begin{table}[h]
\begin{tabular}{|l|l|l|l|l|} \hline

\textbf{Version} & \textbf{Datum} & \textbf{Utförda förändringar} & \textbf{Utförda av} & \textbf{Granskad} \\ \hline
0.1 & - &  Första utkastet & Grupp 4 & - \\ \hline
\end{tabular}
\end{table}

\pagebreak
\pagenumbering{arabic}

\begin{flushleft}

\section{Inledning}
Vid utveckling av en robot som består av olika moduler med olika funktion är det viktigt att undersöka hur dessa moduler ska integreras på bästa sätt. Hur ska dessa moduler kommunicera med varandra och tillsammans funktionera som en robot och inte som delmoduler i sig. För att undersöka detta genomförs en studie av kommunikationsalternativ mellan moduler och hur modulerna på ett bra sätt byggs upp för att lätt kunna integreras med varandra. 

\subsection{Syfte}
Syftet med denna studie är att undersöka hur intern kommunikation och integration av moduler kan ske vid utveckling av en modulär robot samt analysera vad som skulle var lämpligt för vårt projekt.

(Syftet med denna studie är att beröra frågor som rör kommunikation och integration vid modulär uppbyggnad av  för att i projektfasen ha en grund att stå på.)

\pagebreak

\section{Problemformulering}
\textit{Frågeställningar som rapporten ska behandla}

Rapporten ska behandla följande frågor:

\begin{itemize}
	\item Hur fungerar en I\textsuperscript{2}C-buss och varför passar den vårt projekt?
	\item Hur görs integration av moduler på ett smidigt/lättarbetat/bra sätt?
	
		\begin{itemize}
			\item Avbrott eller flera masters?
			\item Parallella bussar?
			\item Hur ska gränssnitt designas?
			\item C eller assembler?
			\item Hur designas tester för att säkerställa att modulen fungerar som den ska?
		\end{itemize}

\end{itemize}

\pagebreak

\section{Kunskapsbas}
\textit{Litteratur, datablad, dokumentation, bakgrundsteori, etc. Dela gärna upp efter vad förstudien handlar om.}

\subsection{I\textsuperscript{2}C - hur microprocessorn fungerar}
En I\textsuperscript{2}C-buss utgörs av två kablar som sköter överföringen mellan två enhter. Varje enhet har en unik adress, och kan konfigureras till att agera både mottagare och sändare. Utöver detta kan enheter kopplade till samma buss hamna i två olika roller, \textit{slave} eller \textit{master}. Denna roll bestämmer vilken/vilka enheter som kan initiera överföring på bussen, vilka är de med rollen master. På varje buss kan det finnas flera master-enheter likväl som flera slave-enheter. Hastigheten med vilken data kan skickas är i standardläge 100 kbit/s. Vid High speed-mode kan komma den upp i 3.4 Mbit/s.

Överföringsprocessen består av fem steg: (1) Initiering, (2) adress till slav, (3) datariktning, (4) informationsutbyte och (5) avslut. (2) och (3) är informationen i den första byten. Viktigt att notera är att när en master-enhetet initierat anses alla andra enheter vara slavar. I (4) måste informationen skickas i hela bytes, där den mest signifikanta biten (MSB) skickas först. Antalet bytes som kan skickas per överföring är inte reglerat.

När bussen är ledig ligger båda ledningarna SDA och SCL höga. (1) och (5) triggas av negativ respektive positiv flank på SDA så länge SCL ligger hög. Ett alternativ till att trigga avslut är att trigga en ny start, dvs att master-enheten har fortsatt kommunikation över bussen. Är inte slaven redo för överföring kan den sätta master-enheten i vänteläge.

https://docs.isy.liu.se/twiki/pub/VanHeden/DataSheets/i2cspec2000.pdf

Ci tänker att vi har en avbrottstyrd i2c va? eller en pollad?


\subsection{Lovisa flummar}



Sätt att utveckla system som innehåller flera moduler.
Kan använda sig av topdown integration eller bottom up ( det vi ska använda iom LIPsmodellen).

Varför använda c? Det tar upp me rminne men ändå kanske det är smidigast? Samtidfigt kan vi assembler bättre och om vi ändå ska tänka assembler och sne översätta till c varför inte bara skriva i assembler? Lättare att se med avbrottsflaggor.
HUr ska vi kunna säkerställa att systemet hålls i realtid och inte alla avbrott hanteras för sent? Avbrottsrutiner extremt korta. Endast det mest nödvändiga där. Hur ha kvar information om avbryten proess vid avbrott?

\subsection{?}
Enligt \cite{GenoM} finns det tre aspekter att ha i åtanke vid uppbyggnad av en robots funktionsnivå: Realtidssystem, kontrollerat system och öppet system. 

\textit{Realtidssystem}- Denna nivå beskriver processerna mellan roboten och dess omgivning genom sensorer exempelvis. Denna nivå behöver kunna utföra alla delar som behöver ske i realtid så som exempelvis kontroll av styrservo och behöver därför kunna utföra synkronisering och kommunicera mellan "uppgifter" för att nämna några saker. 

\textit{Kontrollerat system}- ?

\textit{Öppet system}- Roboten borde vara modifierbar, det ska var lätt att integrera eller ändra olika funktioners utformning efter tilltänkt tillämpning. För att göra detta behövs ett stegvis uppbyggt system med vanlig integrationsmetodologi. 

Alltså behövs en modulärt uppbyggd robot både vad gäller funktionsnivå, standard-uppbyggnad och gränsnitt för modulerna. 


\subsection{Assembler och C för programmering av AVR}

Avbrott och annat.. i assembler
i C

Det finns också möjlighet att kombinera båda språken. I \cite{AssC} beskrivs hur Assembler kan användas i ett C-projekt i Atmel Studio 6. En Assembler rutin kan bli synlig för en C kodad fil och de kan dela globala variabler. Detta kan vara användbart om vissa delar blir enklare eller tydligare att koda i Assembler.

\pagebreak

\section{Rapportens huvuddel, byt rubriknamn}
\textit{Modeller, beräkningar, analyser och ev experiment. Dela gärna upp efter vad förstudien handlar om.}

\subsection {Assembler eller C}

Det finns både fördelar och nackdelar med båda programmeringsspråken. De är båda hårdvarunära språk. C tar som nämnt ovan ofta större minnesutrymme när koden exekveras(?). Detta ses dock inte som något problem då ATmega16 anses ha gott om extra minne.
I projektet som ska genomföras har medlemmarna erfarenhet av Assembler programmering. 
\pagebreak

\section{Resultat och slutsatser}
text

\pagebreak
\addcontentsline{toc}{section}{Referenser}
\bibliographystyle{ieeetr}
\bibliography{references}

\pagebreak
\appendix
\section{First Appendix}

\end{flushleft}
\end{document}