\documentclass[11pt]{article}

\usepackage{extras} % Se extras.sty

\begin{document}
\begin{titlepage}
\begin{center}

{\Large\bfseries TSEA56 - Kandidatprojekt i elektronik \\ LIPS Kappa}

\vspace{5em}

Version 1.0

\vspace{5em}
Grupp 4 \\
\begin{tabular}{rl}
Hynén Ulfsjöö, Olle&\verb+ollul666+
\\
Wasteson, Emil&\verb+emiwa068+
\\
Tronje, Elena&\verb+eletr654+
\\
Gustafsson, Lovisa&\verb+lovgu777+
\\
Inge, Zimon&\verb+zimin415+
\\
Strömberg, Isak&\verb+isast763+
\\
\end{tabular}

\vspace{5em}
\today

\vspace{16em}
Status
\begin{longtable}{|l|l|l|} \hline

Granskad & - & - \\ \hline
Godkänd & - & - \\ \hline
 
\end{longtable}

\end{center}
\end{titlepage}

\pagebreak
\begin{center}

\section*{PROJEKTIDENTITET}
2016/VT, Undsättningsrobot Gr. 4

Linköpings tekniska högskola, ISY
\vspace{5em}
%\begin{center}

\begin{tabular}{|l|l|l|l|} \hline
\textbf{Namn} & \textbf{Ansvar} & \textbf{Telefon} & \textbf{E-post}  \\ \hline 
Isak Strömberg (IS) & Projektledare & 073-980 38 50 & isast763@student.liu.se \\ \hline
Olle Hynén Ulfsjöö (OHU)& Dokumentansvarig & 070-072 91 84 & ollul666@student.liu.se \\ \hline
Emil Wasteson (EW) & Hårdvaruansvarig & 076-836 61 66 & emiwa068@student.liu.se \\ \hline
Elena Tronje (ET) & Mjukvaruansvarig & 072-276 92 93 & eletr654@student.liu.se \\ \hline
Zimon Inge (ZI)& Testansvarig & 070-171 35 18 & zimin415@student.liu.se \\ \hline
Lovisa Gustafsson (LG) & Leveransansvarig & 070-210 32 53 & lovgu777@student.liu.se \\ \hline
\end{tabular}

%\end{center}

E-postlista för hela gruppen: isast763@student.liu.se

\vspace{5em}
Kund: ISY, Linköpings universitet 
tel: 013-28 10 00, fax: 013-13 92 82 \\
Kontaktperson hos kund: Mattias Krysander \\
tel: 013-28 21 98, e-post: matkr@isy.liu.se \\

\vspace{5em}
Kursansvarig:  Tomas Svensson\\
tel: 013-28 13 68, e-post: tomass@isy.liu.se \\
Handledare: Peter Johansson \\
tel: 013-28 13 45, e-post: peter.a.johansson@liu.se
\end{center}
\pagebreak

\tableofcontents

\pagebreak

\section*{Dokumenthistorik}
\begin{table}[h]
\begin{tabular}{|l|l|l|l|l|} \hline

\textbf{Version} & \textbf{Datum} & \textbf{Utförda förändringar} & \textbf{Utförda av} & \textbf{Granskad} \\ \hline
1.0 & - &  Första utkastet & Grupp 4 & - \\ \hline
\end{tabular}
\end{table}

\pagebreak
\pagenumbering{arabic}

\begin{flushleft}
\section{Inledning}
\textit{Ge en översiktlig beskrivning av produkten och uppdraget gärna kopplat till bilder.}
\textit{Lyft gärna fram det som ni anser är utmanande/intressant i uppdraget.}
\textit{Beskriv kortfattat dispositionen på rapporten.}

Projektgruppen har haft som uppdrag att konstruera en autonom undsättningsrobot med kartläggning för utforskning av ett okänt grottsystem. I grottsystemet finns en nödställd som roboten ska hitta och leverera en förnödenhet till. Produkten jämförs med ett antal liknande robotar i en tävling enligt tävlingsregler i appendix \ref{}. Projektdirektivet beskriver närmare projektets förutsättningar. 

Detta dokument beskriver projektets genomförande och dess resultat. Bifogat finns projektdokumentation som skrivits under projektets gång.

\pagebreak

\section{Problemformulering}
\textit{Redogör kort för kravbilden och referera till projektdirektiv och kravspecifikation för att läsa detaljer.}
\textit{Lägg återigen mest fokus på det som ni ansåg utmanande/intressant.}

För att klara uppgiften behöver ett roboten uppfylla ett antal grundläggande krav. Förutom att navigera i grottsystemet och identifiera den nödställde den måste även bestämma kortaste väg mellan ingången och den nödställde. För att kunna göra det delas grottsystemet in i moduler om 40x40 cm som var och en har ett unikt index, baserat på dess x- och y-position.

Även att rita upp en karta som representerar grottsystemets utformning var ett grundkrav. 

%För att klara uppgifen med godkänt resultat fanns det från projektets start uppsatta krav som skulle uppfyllas. Utöver grundkraven sattes även ett antal lägre prioriterade krav upp, vilka skulle uppfyllas i mån av tid men var inte krav för ett godkänt projekt. Nedan följer beskrivning av de mest utmärkande kraven, för en fullständig lista av hög- och lågprioriterade, se appendix \ref{Kravspec}. 

\subsection{Tillräckligt bra sensorvärden att reglera på}
Ett 




\subsection{Kartläggning}

\subsection{Kommunikation}

\subsection{Representation i datormodul}

\subsection{Integration av moduler}




\pagebreak

\section{Kunskapsbas}
\textit{Beskriv kortfattat och referera litteratur, datablad och dokumentation som ni använt er av för att genomföra projektet.}

\pagebreak

\section{Genomförande}

\textit{Beskriv hur projektet har bedrivits och referera bland annat till LIPS-modellen, systemskiss, projektplan och designspecifikation. Observera att designspecifikationen belyser arbetsprocessen och inte den slutliga produkten så om ni uppdaterat designspecifikationen löpande under projektet så kan ni infoga t ex den version som var aktuell vid BP3.}

Projektet har bedrivits enligt projektmodellen \textit{LIPS}-modellen. (referens?) Ett projektdirektiv tilllhandahölls av projektets beställare och var utgångspunkt vid framtagandet av projektets krav. Kraven delades in i tre olika prioriteringsnivåer där prioritet 1 var krav som måste uppfyllas, krav av prioritet 2 som borde uppfyllas i mån av tid och krav av prioritet 3 som kan ses som utvecklingmöjligheter. Kravens prioritering bestämdes utifrån projektdirektivet och projektgruppens intresseområden. 

Efter att kravspecifikationen färdigställts och godkänts av beställare påbörjades den förberedande fasen av projektet. En systemskiss upprättades där produktens övergripande funktionalitet och design beskrevs som ett första förslag på robotens utformning. En detaljerad beskrivning finns i appendix \ref{Systemskiss}. 

Systemskissen låg sedan till grund för arbetet med projektplanen, där projektets aktiviteter togs fram. För att enklare kunna planera projektet identifierades alla aktiviteters föregångare för att tydligt kunna se vilka aktiviteter som behövde slutföras innan en annan kunde påbörjas. Lista över aktiviteter tillsammans med resten av projektplanen återfinns i appendix \ref{Projektplan}. Där finns även tidplanen, vilken beskriver vilka aktiviteter som planerades att genomföras och hur dessa var beroende av varandra. Tidplanen uppdaterades kontinuerligt under projektets gång.

Före projektets praktiska del starade gjordes, i par om två projektmedlemmar, förstudier inom områdena reglering, sensorer och kommunikation. Förstudierna låg sedan till grund för designspecifikationen som ger en ingående beskrivning av produkten. Syftet med designspecifikationen är att använda den som underlag vid det praktiska arbetet. Den skulle vara så detaljerad att all funktionalitet samt hårdvaru-och mjukvarudesign skulle vara tillräckligt utförligt beskriven för att kunna arbeta utefter dokumentet. Designspecifikationen i sin helhet hittas i appendix \ref{Designspecifikation}. 

När designspecifikationen var färdigställt började det praktiska arbetet. Eftersom roboten hade tre tydligt förutbestämda delsystem föll det sig naturligt att det par som arbetat med respektive förstudie bar huvudansvar och jobbade mest med just det delsystemet hos roboten. Mjukvara för busskommunikation var något som tidigt färdigställdes eftersom arbetet med integreration av moduler skulle kunna påbörjas så tidigt som möjligt. Även avläsning av sensorer var också något som tidigt började arbetas med eftersom regleringen var beroende av att erhålla korrekta sensorvärden. För att mjukvaran skulle kunna testas så tidigt som möjligt påbörjades även virning av hårdvara tidigt. 

Arbetet följde tidplanens aktiviteter i den förutbestämda ordning de behöbvde utföras i. Som nämnt uppdaterades tidplanen allt eftersom aktiviteter avklarades för att bättre visa det faktiska arbetets gång. Till hjälp vid utvecklingsarbetet av mjukvara till hårdvaran användes Atmel Studio. Mjukvaran till datorn utvecklades i Java. 

\pagebreak

\section{Teknisk beskrivning}
\textit{Redovisa det tekniska resultatet och förstudierna på ett lämpligt sätt. Detaljerade beskrivningar på grindnivå i hårdvaran eller på instruktionsnivå i mjukvaran är i detta sammanhang oftast ointressant. Referera den tekniska dokumentationen och förstudierna för beskrivning av detaljer. Lyft fram egna kreativa lösningar!}
\pagebreak

\section{Resultat}
\textit{Beskriv kortfattat hur produkten används, hänvisa till användarmanualen.}
\textit{Beskriv vilken prestanda produkten har och hur det har testats, hänvisa till eventuella testprotokoll.}
\textit{Kommentera om produkten klarade de kraven som ställdes upp i kravspecifikationen. }

\pagebreak

\section{Slutsatser}
\textit{Slutsatser
Sammanfatta arbetet. 
Lyft fram det ni är mest nöjda med.
Reflektera över resultatet såväl tekniskt som över genomförandet. Referera efterstudien.}
\textit{Framtida arbete
Vad skulle ni göra annorlunda om ni skulle göra om samma uppdrag?
Vad skulle ni vilja utveckla om ni fick mer tid?
Hur skulle ni tänka er att ändra uppgiften för att göra den ännu mer intressant?}

\pagebreak

\section{Referenser}
\textit{Ange de referenser som hänvisas till från kappan. Nämn gärna också att fler referenser finns i bilagorna.}

\pagebreak
\addcontentsline{toc}{section}{Referenser}
%\bibliographystyle{ieeetr}
%\bibliography{references}

\pagebreak


\appendix

\end{flushleft}

\end{document}


