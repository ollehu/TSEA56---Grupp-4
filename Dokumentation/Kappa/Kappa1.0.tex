\documentclass[11pt]{article}

\usepackage{extras} % Se extras.sty

\begin{document}
\begin{titlepage}
\begin{center}

{\Large\bfseries TSEA56 - Kandidatprojekt i elektronik \\ LIPS Kappa}

\vspace{5em}

Version 1.0

\vspace{5em}
Grupp 4 \\
\begin{tabular}{rl}
Hynén Ulfsjöö, Olle&\verb+ollul666+
\\
Wasteson, Emil&\verb+emiwa068+
\\
Tronje, Elena&\verb+eletr654+
\\
Gustafsson, Lovisa&\verb+lovgu777+
\\
Inge, Zimon&\verb+zimin415+
\\
Strömberg, Isak&\verb+isast763+
\\
\end{tabular}

\vspace{5em}
\today

\vspace{16em}
Status
\begin{longtable}{|l|l|l|} \hline

Granskad & - & - \\ \hline
Godkänd & - & - \\ \hline
 
\end{longtable}

\end{center}
\end{titlepage}

\pagebreak
\begin{center}

\section*{PROJEKTIDENTITET}
2016/VT, Undsättningsrobot Gr. 4

Linköpings tekniska högskola, ISY
\vspace{5em}
%\begin{center}

\begin{tabular}{|l|l|l|l|} \hline
\textbf{Namn} & \textbf{Ansvar} & \textbf{Telefon} & \textbf{E-post}  \\ \hline 
Isak Strömberg (IS) & Projektledare & 073-980 38 50 & isast763@student.liu.se \\ \hline
Olle Hynén Ulfsjöö (OHU)& Dokumentansvarig & 070-072 91 84 & ollul666@student.liu.se \\ \hline
Emil Wasteson (EW) & Hårdvaruansvarig & 076-836 61 66 & emiwa068@student.liu.se \\ \hline
Elena Tronje (ET) & Mjukvaruansvarig & 072-276 92 93 & eletr654@student.liu.se \\ \hline
Zimon Inge (ZI)& Testansvarig & 070-171 35 18 & zimin415@student.liu.se \\ \hline
Lovisa Gustafsson (LG) & Leveransansvarig & 070-210 32 53 & lovgu777@student.liu.se \\ \hline
\end{tabular}

%\end{center}

E-postlista för hela gruppen: isast763@student.liu.se

\vspace{5em}
Kund: ISY, Linköpings universitet 
tel: 013-28 10 00, fax: 013-13 92 82 \\
Kontaktperson hos kund: Mattias Krysander \\
tel: 013-28 21 98, e-post: matkr@isy.liu.se \\

\vspace{5em}
Kursansvarig:  Tomas Svensson\\
tel: 013-28 13 68, e-post: tomass@isy.liu.se \\
Handledare: Peter Johansson \\
tel: 013-28 13 45, e-post: peter.a.johansson@liu.se
\end{center}
\pagebreak

\tableofcontents

\pagebreak

\section*{Dokumenthistorik}
\begin{table}[h]
\begin{tabular}{|l|l|l|l|l|} \hline

\textbf{Version} & \textbf{Datum} & \textbf{Utförda förändringar} & \textbf{Utförda av} & \textbf{Granskad} \\ \hline
1.0 & - &  Första utkastet & Grupp 4 & - \\ \hline
\end{tabular}
\end{table}

\pagebreak
\pagenumbering{arabic}

\begin{flushleft}
\section{Inledning}
\textit{Ge en översiktlig beskrivning av produkten och uppdraget gärna kopplat till bilder.}
\textit{Lyft gärna fram det som ni anser är utmanande/intressant i uppdraget.}
\textit{Beskriv kortfattat dispositionen på rapporten.}

Projektgruppen har haft som uppdrag att konstruera en autonom undsättningsrobot med kartläggning för utforskning av ett okänt grottsystem. Produkten jämförs med ett antal liknande robotar i en tävling enligt  tävlingsregler i appendix\ref{}. Projektdirektivet beskriver närmare projektets förutsättningar. 

Detta dokument beskriver projektets genomförande och dess resultat. Bifogat finns projektdokumentation som skrivits under projektets gång.

\pagebreak

\section{Problemformulering}
\textit{Redogör kort för kravbilden och referera till projektdirektiv och kravspecifikation för att läsa detaljer.}
\textit{Lägg återigen mest fokus på det som ni ansåg utmanande/intressant.}

För att klara uppgifen med godkänt resultat fanns det från projektets start uppsatta krav som skulle uppfyllas. Utöver grundkraven sattes även ett antal lägre prioriterade krav upp. Dessa skulle uppfyllas i mån av tid men var inte krav för ett godkänt projekt. Till de högre prioriterade kraven tillhörde att roboten skulle kunna navigera autonomt i ett grottsystem, utföra 90- och 180-graderssvängar samt identifiera den nödställde och den kortaste vägen till denne.

  För alla krav som roboten skulle uppfylla se appendix \ref{}
\pagebreak

\section{Kunskapsbas}
\textit{Beskriv kortfattat och referera litteratur, datablad och dokumentation som ni använt er av för att genomföra projektet.}

\pagebreak

\section{Genomförande}

\textit{Beskriv hur projektet har bedrivits och referera bland annat till LIPS-modellen, systemskiss, projektplan och designspecifikation. Observera att designspecifikationen belyser arbetsprocessen och inte den slutliga produkten så om ni uppdaterat designspecifikationen löpande under projektet så kan ni infoga t ex den version som var aktuell vid BP3.}


Projektet har bedrivits enligt projektmodellen \textit{LIPS}-modellen. (referens?) Ett projektdirektiv tilllhandahölls av projektets beställare och denna var utgångspunkt vid framtagandet av projektets krav. Kraven delades in i tre olika prioriteringsnivåer där prioritet 1 var krav som måste uppfyllas, prioritet 2 krav som borde uppfyllas i mån av tid och prioritet 3 krav som kunde ses som utvecklingmöjligheter. Kravens prioritering bestämdes utifrån projektdirektivet och projektgruppens intresseområden. 

Efter kravspecifikationen färdigställts och godkänts av beställare påbörjades den förberedande fasen av projektet för planering av detta. En systemskiss upprättades där produktens övergripande funktionalitet och design beskrevs som ett första förslag på robotens utformning. För en bättre förståelse se \ref{} Systemskissen låg till grund för arbetet med projektplanen där projektets aktiviteter togs fram. 
\pagebreak

\section{Teknisk beskrivning}
\textit{Redovisa det tekniska resultatet och förstudierna på ett lämpligt sätt. Detaljerade beskrivningar på grindnivå i hårdvaran eller på instruktionsnivå i mjukvaran är i detta sammanhang oftast ointressant. Referera den tekniska dokumentationen och förstudierna för beskrivning av detaljer. Lyft fram egna kreativa lösningar!}
\pagebreak

\section{Resultat}
\textit{Beskriv kortfattat hur produkten används, hänvisa till användarmanualen.}
\textit{Beskriv vilken prestanda produkten har och hur det har testats, hänvisa till eventuella testprotokoll.}
\textit{Kommentera om produkten klarade de kraven som ställdes upp i kravspecifikationen. }

\pagebreak

\section{Slutsatser}
\textit{Slutsatser
Sammanfatta arbetet. 
Lyft fram det ni är mest nöjda med.
Reflektera över resultatet såväl tekniskt som över genomförandet. Referera efterstudien.}
\textit{Framtida arbete
Vad skulle ni göra annorlunda om ni skulle göra om samma uppdrag?
Vad skulle ni vilja utveckla om ni fick mer tid?
Hur skulle ni tänka er att ändra uppgiften för att göra den ännu mer intressant?}

\pagebreak

\section{Referenser}
\textit{Ange de referenser som hänvisas till från kappan. Nämn gärna också att fler referenser finns i bilagorna.}

\pagebreak
\addcontentsline{toc}{section}{Referenser}
\bibliographystyle{ieeetr}
\bibliography{references}

\pagebreak


\appendix

\end{flushleft}

\end{document}

