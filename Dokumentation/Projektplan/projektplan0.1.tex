\documentclass[11pt]{article}

\usepackage{extras} % Se extras.sty

%\titlespacing\section{0pt}{12pt plus 4pt minus 2pt}{0pt plus 2pt minus 2pt}
%\titlespacing\subsection{0pt}{12pt plus 4pt minus 2pt}{0pt plus 2pt minus 2pt}
%\titlespacing\subsubsection{0pt}{12pt plus 4pt minus 2pt}{0pt plus 2pt minus 2pt}

\begin{document}
\begin{titlepage}
\begin{center}

{\Large\bfseries TSEA56 - Kandidatprojekt i elektronik \\ LIPS Projektplan}

\vspace{5em}

Version 0.1

\vspace{5em}
Grupp 4 \\
\begin{tabular}{rl}
Hynén Ulfsjöö, Olle&\verb+ollul666+
\\
Wasteson, Emil&\verb+emiwa068+
\\
Tronje, Elena&\verb+eletr654+
\\
Gustafsson, Lovisa&\verb+lovgu777+
\\
Inge, Zimon&\verb+zimin415+
\\
Strömberg, Isak&\verb+isast763+
\\
\end{tabular}

\vspace{5em}
\today

\vspace{16em}
Status
\begin{longtable}{|l|l|l|} \hline

Granskad & - & - \\ \hline
Godkänd & - & - \\ \hline
 
\end{longtable}

\end{center}
\end{titlepage}

\pagebreak
\begin{center}

\section*{PROJEKTIDENTITET}
2016/VT, Undsättningsrobot Gr. 4

Linköpings tekniska högskola, ISY
\vspace{5em}
\begin{center}

\begin{tabular}{|l|l|l|l|} \hline
\textbf{Namn} & \textbf{Ansvar} & \textbf{Telefon} & \textbf{E-post}  \\ \hline 
Isak Strömberg (IS) & Projektledare & 073-980 38 50 & isast763@student.liu.se \\ \hline
Olle Hynén Ulfsjöö (OHU)& Dokumentansvarig & 070-072 91 84 & ollul666@student.liu.se \\ \hline
Emil Wasteson (EW) & Hårdvaruansvarig & 076-836 61 66 & emiwa068@student.liu.se \\ \hline
Elena Tronje (ET) & Mjukvaruansvarig & 072-276 92 93 & eletr654@student.liu.se \\ \hline
Zimon Inge (ZI)& Testansvarig & 070-171 35 18 & zimin415@student.liu.se \\ \hline
Lovisa Gustafsson (LG) & Leveransansvarig & 070-210 32 53 & lovgu777@student.liu.se \\ \hline
\end{tabular}

\end{center}

E-postlista för hela gruppen: isast763@student.liu.se

\vspace{5em}
Kund: ISY, Linköpings universitet \\
tel: 013-28 10 00, fax: 013-13 92 82 \\
Kontaktperson hos kund: Mattias Krysander \\
tel: 013-28 21 98, e-post: matkr@isy.liu.se \\

\vspace{5em}
Kursansvarig:  Tomas Svensson\\
tel: 013-28 13 68, e-post: tomass@isy.liu.se \\
Handledare: Peter Johansson \\
tel: 013-28 13 45, epost: peter.a.johansson@liu.se
\end{center}
\pagebreak

\tableofcontents

\pagebreak

\section*{Dokumenthistorik}
\begin{table}[h]
\begin{tabular}{|l|l|l|l|l|} \hline

\textbf{Version} & \textbf{Datum} & \textbf{Utförda förändringar} & \textbf{Utförda av} & \textbf{Granskad} \\ \hline
0.1 & FYLL I &  Första utkastet & Grupp 4 & - \\ \hline
\end{tabular}
\end{table}

\pagebreak
\pagenumbering{arabic}

\begin{flushleft}
\section{Beställare}
Beställare är Mattias Krysander, ISY Fordonssystem.

\section{Översiktlig beskrivning av projektet}
Nedan följer en grov beskrivning av projektet.
\subsection{Syfte och mål}
Syftet med projektet är att utveckla en undsättningsrobot som ska kunna leta upp nödställda i ett labyrintsystem för att sedan åka och hämta relevant nödutrustning. För att klara detta ska roboten kunna manövrera i små utrymmen. 

Projektets mål är att maximera överlevnadschansen för de nödställda genom att på snabbast möjliga tid förse de nödställda med utrustning, alltså hitta kortast möjliga väg från ingång till nödställda.

Gruppens mål med projektet är att få en större förståelse för hur olika system integrerar med varandra, inte bara hur delsystem fungerar när de är separerade.


\subsection{Leveranser}
Leveranserna i projektet ska ske enligt nedanstående lista.
\begin{center}
\begin{longtable}{|p{.01\linewidth} p{.07\linewidth} | p{.7\linewidth} | p{.1\linewidth} |} \hline
\textbf{Datum} & & \textbf{Aktivitet} & \textbf{BP} \\ \hline
9 & feb & Val av förstudier ska vara inlämnade till beställaren. & - \\ \hline
15 & feb & Första versionen av projektplan, tidplan och systemskiss ska vara inlämnade till beställaren. & - \\ \hline
19 & feb & Slutgiltlig version av projektplan, tidplan och systemskiss ska vara inlämnade till beställaren.& BP2 \\ \hline
3 & mar & Första versionen av förstudien ska vara inlämnad till handledaren och beställaren. & - \\ \hline
11 & mar & Första versionen av designspecifikationen ska vara inlämnad till handledaren. & - \\ \hline
5 & apr & Designspecifikationen ska vara godkänd av handledaren. & BP3 \\ \hline
8 & apr & Version 1.0 av förstudien ska vara inlämnad till handledaren och beställaren. & - \\ \hline
15 & apr & Design ska vara presenterad och godkänd av handledaren. & BP4 \\ \hline
19 & maj & Version 1.0 av \textit{Kappan} (exklusive appendix) ska vara inlämnad. & - \\ \hline
25 & maj & Kraven ska vara verifierade.  & BP5 \\ \hline
26 & maj & Version 1.0 av teknisk dokumentation och användarhandledning ska vara inlämnade till beställaren. & - \\ \hline
\multicolumn{2}{| l |}{vecka  22} &  Redovisning och demostration. & - \\ \hline
3 & jun & Efterstudie och källskod ska vara inlämnade. & - \\ \hline
10 & jun & All utrsutning och nycklar ska vara återlämnade. & - \\ \hline
\end{longtable}
\end{center}

Utöver ovanstående leveranser ska tidsrapporter lämnas senaste kl 16.00 vid följande datum: 3 februari, 22 februari, 7 mars, 14 mars, 4 april, 11 april, 18 april, 25 april, 2 maj, 9 maj, 16 maj, 23 maj, 30 maj och 7 juni. 

\subsection{Begränsningar}
Roboten kommer vara begrnsad till att röra sig i korridorer à 40 cm och kommer inte kunna manövreras i öppna rum. Sensorerna kommer att vara begränsade till att enbart kunna detektera ..... 

\pagebreak
\section{Fasplan}
Nedan följer en översiktlig beskrivning av projektets faser.
\subsection{Före projektstart}
I fasen före projektstart tas en kravspecifikation fram i samråd med beställaren. Denna specifikation har sin utgång i projektdirektivet gruppen fått ut. En utförlig tidsplan ska tas fram för att ge en översiktlig bild av vad som ska göras och hur mycket tid som ska läggas under vilka veckor. Före det praktiska arbetet inleds ska förstudier på alla delssystem göras så projektgruppen erhåller rätt kunskaper för att genomföra projektet. Även en systemskiss och projektplan ska tas fram före projektets start.

\subsection{Under projektet}
Till att börja med görs en designspecifikation för att underlätta inför konstruktionsmomenten av roboten. Delsystemen konstrueras och tester sker för att kontrollera att de uppfyller kravspecifikationen. Alla tester dokumenteras. Om kravspecifikationen ej uppfylls ska möte med beställaren bokas för att se över eventuella ändringar i denna. Under projektets gång ska avstämning mot tid- och projektplan ske kontinuerligt för att se till att planerna efterföljs.

\subsection{Efter projektet}
När roboten uppfyller alla krav i kravspecifikationen levereras den till beställaren. Därefter ska efterstudier göras för att analysera och utvärdera projektet. Efter projektet genomförs även en tävling mot grupper som har genomfört samma projekt för att se vilken grupp som har lyckats bäst. 

\pagebreak
\section{Organisationsplan för hela projektet}
Projektet kommer bedrivas enligt hierarkin i figur \ref{hierarki}.

\begin{figure}[htbp]
\centering
\noindent\resizebox{.5\linewidth}{!}{
	\documentclass[crop,tikz]{standalone}
\usepackage{tikz}
\usetikzlibrary{calc}
\usetikzlibrary{positioning}
\begin{document}
\begin{tikzpicture}
%\footnotesize

\tikzset{every node/.style={thick, draw=black, align=center, minimum height=40pt, text width=100pt, minimum width=100pt}}

\node(bestallare) {\textbf{Beställare} \\ \vspace{2mm} Mattias Krysander};

\node[below=20pt] (projektgrupp) at (bestallare.south) {\textbf{Projektgrupp} \\ \vspace{2mm} Grupp 4};

\node[above right=-15pt and 40pt of projektgrupp] (handledare) {\textbf{Handledare} \\ \vspace{2mm} Peter Johansson};
\node[below right=-15pt and 40pt of projektgrupp] (experter) {\textbf{Experter}};

\node[below left = 50pt and -35pt of projektgrupp] (pm3) {\textbf{Hårdvaruansvarig} \\ \vspace{2mm} Emil Wasteson};
\node[below right = 50pt and -35pt of projektgrupp] (pm4) {\textbf{Mjukvaruansvarig} \\ \vspace{2mm} Elena Tronje};
\node[below = 20pt of pm3] (pm2) {\textbf{Dokumentansvarig} \\ \vspace{2mm} Olle Hynén Ulfsjöö};
\node[below = 20pt of pm2] (pm1) {\textbf{Projektledare} \\ \vspace{2mm} Isak Strömberg};
\node[below = 20pt of pm4] (pm5) {\textbf{Testansvarig} \\ \vspace{2mm} Zimon Inge};
\node[below = 20pt of pm5] (pm6) {\textbf{Leveransansvarig} \\ \vspace{2mm} Lovisa Gustafsson};

\coordinate (membertop1) at ($(projektgrupp.south)+(0,-70pt)$);
\coordinate (membertop2) at ($(membertop1)+(0,-60pt)$);
\coordinate (membertop3) at ($(membertop2)+(0,-60pt)$);
\coordinate(helptop) at ($(projektgrupp.east)+(30pt,0)$);

\draw[thick] (projektgrupp) -- (membertop1)
	(pm1.east) -| (membertop2) |- (pm6.west)
	(pm2.east) -| (membertop1) |- (pm5.west)
	(pm3.east) -| (membertop1) |- (pm4.west)
	
	(projektgrupp.east) -- (helptop)
	(handledare.west) -| (helptop) |- (experter.west)
	
	(bestallare.south) -- (projektgrupp.north);
\end{tikzpicture}
\end{document}}
	\caption{Projekthierarki} \label{hierarki}	
\end{figure}

\subsection{Villkor för samarbete inom projektgtuppen}
\begin{itemize}
	\item Målsättning är att ha en fast mötestid i veckan då hela projektgruppen träffas. Vid mötet ska en veckoplan tas fram.
	\item Gruppen eftersträvar att träffas i helgrupp ytterligare en gång i veckan för idéutbyte och gemensamt arbete.
	\item Om inte gruppen eller projektledaren känner sig missnöjda ska nuvarande projektledare inneha rollen under hela projektets gång.
	\item De aktiviteter som bestäms på veckomötet ska ha en ytterst ansvarig person, oavsett hur många som arbetar med uppgiften.
	\item Ansvarig gruppmedlem för en uppgift ska varken arbeta mer eller mindre än övriga i gruppen, enbart se till att uppgiften klaras av inom tidsramen.
\end{itemize}

\subsection{Definition av arbetsinnehåll och ansvar}

\begin{longtable}{| p{.2\linewidth} | p{.2\linewidth} | p{.5\linewidth} |} \hline
\textbf{Namn} & \textbf{Projektroll} & \textbf{Ansvar} \\ \hline \endhead

\multicolumn{3}{r@{}}{fortsättning \ldots} \\
\endfoot
\endlastfoot
Isak Strömberg & Projektledare & \vspace{-\baselineskip}
\begin{itemize}[label={--},leftmargin=*,nosep]
\item Övergripande ansvar för projektet.
\item Säkerställer att ingen dubbelarbete görs.
\item Bokar lokal och skickar kallelse till möten.
\item Fördela ut ansvar för arbetsområdena. 
\item Håller kontakt med beställare.
\vspace{-\baselineskip}
\end{itemize} 
\\ \hline

Olle Hynén Ulfsjöö & Dokumentansvarig & \vspace{-\baselineskip}
\begin{itemize}[label={--},leftmargin=*,nosep]
\item Ansvarar för protokoll på möten.
\item Ansvarar för dokumentation och layout.
\item Står för sista gransning av dokument.
\vspace{-\baselineskip}
\end{itemize}
\\ \hline

Emil Wasteson & Hårdvaruansvarig & \vspace{-\baselineskip}
\begin{itemize}[label={--},leftmargin=*,nosep]
\item Ansvarar för kopplingsschema.
\item Ansvarar för beställning av hårdvara.
\vspace{-\baselineskip}
\end{itemize}
\\ \hline

Elena Tronje & Mjukvaruansvarig & \vspace{-\baselineskip}
\begin{itemize}[label={--},leftmargin=*,nosep]
\item Ansvarar för att kod skrivs i lämpligt språk.
\item Utför granskning av all kod som skrivs.
\item Ansvar över modulernas kommunikation.
\vspace{-\baselineskip}
\end{itemize}
\\ \hline

Zimon Inge & Testansvarig & \vspace{-\baselineskip}
\begin{itemize}[label={--},leftmargin=*,nosep]
\item Ansvarar för slutgiltliga tester.
\item Ser till att testprotokoll förs.
\vspace{-\baselineskip}
\end{itemize}
\\ \hline

Lovisa Gustafsson & Leveransansvarig & \vspace{-\baselineskip}
\begin{itemize}[label={--},leftmargin=*,nosep]
\item Ser till att material levereras innan deadline.
\item Fördelar extraresurser.
\item Påminner gruppen om deadlines.
\vspace{-\baselineskip}
\end{itemize}
\\ \hline

\end{longtable}

\pagebreak
\section{Dokumentplan}
Följande tabell presenterar de olika dokument som ingår i projektet. 

\begin{longtable}{| p{.17\linewidth} | p{.11\linewidth} | p{.31\linewidth} | p{.15\linewidth} | p{.11\linewidth} |} \hline
\textbf{Dokument} & \textbf{Ansvarig} & \textbf{Syfte} & \textbf{Mottagare} & \textbf{Deadline} \\ \hline
Krav-specifikation & Alla & \textit{Definierar alla krav på systemet.} & Beställare och projektgrupp & 2 feb \\ \hline
Projektplan & Alla & \textit{Specificerar projektets upplägg.} & Beställare och projektgrupp & 19 feb \\ \hline
Tidplan & Alla & \textit{Planerar arbetsinsatsen.} & Beställare och projektgrupp & 19 feb \\ \hline
Systemskiss & Alla & \textit{Beskriver produktens upplägg.} & Beställare och projektgrupp & 19 feb \\ \hline
Förstudie & Alla & \textit{Utforskar de tekniska alternativ som finns tillgängliga.} & Beställare, projektgrupp & 8 apr \\ \hline
Design-specifikation & Alla & \textit{Ger en mer detaljerad beskrivning av produkten.} & Beställare, projektgrupp och hand-ledare & 5 apr \\ \hline
Kappa & Alla & \textit{Sammanfattar projektet.} & Beställare & 19 maj \\ \hline
Teknisk \mbox{dokumentation} & Alla & \textit{Beskriver tekniken bakom produkten.} & Beställare & 26 maj\\ \hline
Användar-handledning & Alla & \textit{Beskriver hur produkten används.} & Beställare & 26 maj \\ \hline
Efterstudie & Alla & \textit{Reflekterar över projektet.} & Projektgrupp & 3 jun \\ \hline

\end{longtable}

\pagebreak
\section{Utvecklingsmetodik}
Projektet kommer att genomföras enligt LIPS-modellen. Det tekniska arbetet kommer att ske i grupper om två personer. Programmering kommer i första hand att ske i C, andra språk kommer att användas vid behov. Arbetet med mjukvaran kan ske på egen hand.

\pagebreak
\section{Utbildningsplan}
Eftersom inte tillräckligt med tekniska kunskaper finns hos gruppens medlemmar när projektet inleds kommer samtliga i gruppen behöva utbildas inom de områden där kunskap saknas. 
\subsection{Egen utbildning}
Varje individ i gruppen ansvarar för sin egen utbildning och för att ha de kunskaper som krävs för att genomföra projektet. Dessa timmar kommer att plockas från en \textit{utbildningspool} som innefattar \textcolor{red}{x} antal timmar. Utbildningsområde beror på medlemmens ansvar och består i grunden av C, Git och \LaTeX. 


\pagebreak
\section{Rapporteringsplan}
Alla gruppmedlemmar ska på gruppmötet varje vecka uppdatera övriga i gruppen vad som har gjorts sedan förra mötet. Projektledaren sammanfattar vad som har gjorts och skickar veckovis in en statusrapport, tillsammans med en tidrapport, till beställaren. Tidrapporten utgår från ett exceldokument som fylls i kontinuerligt av gruppens medlemmar.

\pagebreak
\section{Mötesplan}
Ett stående lunchmöte á 45 minuter ska hållas varje onsdag. På mötet ska alla medlemmar delta om ingen speciell anledning finns för att inte kunna medverka. Inför mötet bokar projektledaren en lokal och lägger upp en dagordning i gruppens gemensamma filutrymme. Under mötet för dokumentansvarig protokoll på det som diskuteras/bestäms. Det ska även bestämmas vad varje gruppmedlem ska göra och ha ansvar för under kommande vecka

\pagebreak
\section{Resursplan}
\subsection{Personer}
Projektgruppen består av sex personer där alla lägger samma mängd timmar på projektet. Tekniska experter och en handledare finns att tillgå under projektets gång.

\subsection{Material}
ISY tillhandahåller den hårdvara (sensorer, motorer, etc.) som tillhör projektet. Det finns vid behov även möjlighet att beställa viss utrustning som i dagsläget inte finns hos ISY. 

\subsection{Lokaler}
När designskissen har godkänts av beställaren kommer tillträde ges till ISY:s laborationslokal Muxen, där två stationer kommer finnas till gruppens förfogande. För att få plats kommer det aldrig vistas mer än två grupper om två personer vid stationerna vid samma tillfälle.

\subsection{Ekonomi}
Efter projektplanen har blivit godkänd har gruppen 1380 timmar till sitt förfogande (230 timmar per person).

\pagebreak
\section{Milstolpar och beslutspunkter}
\subsection{Milstolpar}
Nedan följer projektets milstolopar:

\begin{longtable}{| p{.05\linewidth} | p{.7\linewidth} | p{.15\linewidth} |} \hline
\textbf{Nr.} & \textbf{Beskrivning} & \textbf{Datum} \\ \hline
%\milstolpe & Kravspecifikationen är klar & 2016-02-02 \\ \hline
\milstolpe & Designspecifikationen är klar & 2016-04-05 \\ \hline
\milstolpe & Huvudmodulens blåtandskommunikation med PC är klar & 2016-xx-yy \\ \hline
\milstolpe & Sensorenheten (inkl. kommunikation med huvudenheten) är testad och klar & 2016-xx-yy \\ \hline
\milstolpe & Reglering av roboten i rak korridor och sväng är testad och klar & 2016-xx-yy \\ \hline
\milstolpe & Manuell styrning av roboten är testad och klar & 2016-xx-yy \\ \hline
\milstolpe & Beräkningsalgoritm för kortaste väg är klar & 2016-xx-yy \\ \hline
\milstolpe & 2D-karta kan ritas upp på PC & 2016-xx-yy \\ \hline
\milstolpe & Identfiering av nödställda fungerar & 2016-xx-yy \\ \hline


\end{longtable}

\subsection{Beslutspunkter}
Nedan följer en tabell över de beslutspunkter projektet kommer innehålla:

\begin{longtable}{| p{.05\linewidth} | p{.7\linewidth} | p{.15\linewidth} |} \hline
\textbf{Nr.} & \textbf{Beskrivning} & \textbf{Datum} \\ \hline
0 & Godkännande av projektdirektiv & 2016-01-22 \\ \hline
1 & Godkännande av kravspecifikation & 2016-02-02 \\ \hline
2 & Godkännande av projektplan, tidplan och systemskiss & 2016-02-19 \\ \hline
3 & Godkännande av designspecifikation & 2016-04-05 \\ \hline
4 & Konstruktionsgranskning & 2016-04-15 \\ \hline
5 & Verifiering av uppfylld kravspecifikationens & 2016-05-25 \\ \hline
6 & Godkännande av projektet & 2016-06-03 \\ \hline

\end{longtable}

\pagebreak
\section{Aktiviteter}
text

\begin{longtable}{| p{.05\linewidth} | p{.15\linewidth} | p{.4\linewidth} | p{.15\linewidth} | p{.1\linewidth} |} \hline
\textbf{Nr.} & \textbf{Aktivitet} & \textbf{Beskrivning} & \textbf{Föregångare} & \textbf{Tid [h]} \\ \hline
1 & Testplanering & Utarbeta testplan enligt standard IEEE730 & 2 15 & 20 \\ \hline

\end{longtable}

\pagebreak
\section{Tidplan}
Se Appendix A

\pagebreak
\section{Kvalitetsplan}
\subsection{Granskningar}
Alla dokument som skrivs ska granskas av en gruppmedlem som inte har medverkat i framtagandet av dokumentet. Detta gäller för alla typer av dokument, inklusive kod och presentationer. De personer som granskar dokumenten ska enbart komma med kommentarer, sedan är det upp till den som skriver dokumentet att göra justeringar. Detta upprepas till både skrivare och granskare är nöjda.

\subsection{Testplan}
text

\pagebreak
\section{Prioriteringar}
Vid tidsbrist kommer de funktioner som är mest centrala för robotens funktion få högsta prioritet. Detta innebär förutom alla krav på nivå två, funktioner som:
\begin{itemize}
	\item PC-programvarans grafiska gränssnitt... 
	\begin{itemize}	
		\item ...för presentation av data
		\item ...för uppritning av kartan
	\end{itemize}
	\item Optimering av algoritmer i robotens programvara för...
	\begin{itemize}
		\item ...avsökning av labyrinten
		\item ...beräkning av närmsta vägen
		\item ...reglering av robotens körning
	\end{itemize}
	\item Kretskort (i stället för virkort) och andra konstruktionsmässiga robusthetsförbättringar.
\end{itemize}

\pagebreak
\section{Projektavslut}
Projektet avslutas med att en färdig produkt lämnas till beställaren, vilket innebär att alla krav med prioritet 1 är uppfyllda. Avstämning sker mot kravspecifikationen för att försäkra sig om detta. Har krav av prioritet 2 och 3 uppyllt ska detta anges till beställaren och även testas. Som ett test på hur bra projektet har lyckats kommer en tävling mot andra grupper som genomfört projekt med samma direktiv. Det kommer även skrivas en slutrapport och göras en efterstudie. 

\end{flushleft}

\end{document}