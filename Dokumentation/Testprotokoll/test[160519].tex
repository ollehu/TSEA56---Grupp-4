\documentclass[11pt]{article}

\usepackage{extras} % Se extras.sty

\begin{document}

\begin{center}

{\Huge\bfseries Testprotokoll}
\vspace{4em}
\end{center}

\begin{flushleft}


\begin{figure}[htbp]
\centering
\noindent\resizebox{.4\linewidth}{!}{
	\documentclass[border=10px]{standalone}
\usepackage{tikz}
\usetikzlibrary{patterns}
\usetikzlibrary{shapes.arrows}
\usepackage{amssymb}
\begin{document}
	
\begin{tikzpicture}[scale=1]
\draw[thick]  (3,0) -- (4,0);	
\draw[thick]  (8,3) -- (8,4);	
\draw[thick]  (1,7) -- (4,7);	

\draw[thick,pattern=north west lines, pattern color=black]	 	(0,0) -- (3,0) -- (3,1) -- (1,1) -- (1,7) -- (0,7) -- (0,0);
\draw[thick,pattern=north west lines, pattern color=black]	 	(4,0) -- (4,3) -- (8,3) -- (8,0) -- (4,0);
\draw[thick,pattern=north west lines, pattern color=black]	 	(2,2) -- (2,3) -- (3,3) -- (3,2) -- (2,2);
\draw[thick,pattern=north west lines, pattern color=black]	 	(2,4) -- (2,6) -- (3,6) -- (3,4) -- (2,4);
\draw[thick,pattern=north west lines, pattern color=black]	 	(4,4) -- (4,7) -- (8,7) -- (8,4) -- (4,4);

\node at (1.5,1.5)[color=red] {X};
\node at (2.5,1.5)[color=blue] {1};
\node at (3.5,1.5)[color=blue] {3};
\node at (3.5,2.5)[color=blue] {2};
\node at (3.5,0.5)[color=blue] {4};

\draw[thick,->]  (7.8,3.5) -- (7.3,3.5);	
	\end{tikzpicture}
	
\end{document}}
	\caption{Översikt av testbana\label{bana}}	
\end{figure}

\begin{description}
\item[Vad som har testats] \hfill \\
Det har i tre omgångar testats huruvida avsökningsagoritmen kan hantera att målet är placerat i ett hörn.

\vspace{1em}
\item[Utfall 1] \hfill \\
I en första körning placerade den ut väggar i sin karta på de positioner markerade med 1, 2, 3 och 4 i figur \ref{bana}. Till följd av detta stannaden den där utan att säga att den var klar.

Efter detta justerades det i hur värden i den interna kartan uppdateras då målets ruta markeras med en siffra skild från de som används för uppräkningen.

\item[Utfall 2] \hfill \\
I andra körningen blev kartan rätt men roboten stannade i den modul markerad med 1 i figur \ref{bana} utan att säga att den är klar med avsökningen.

För att lösa detta justerades det så att det är okej att köra in i den ruta som målet står i då implementationen byggt på att det inte varit okej att ställa sig i den ruta som målet är i.

\item[Utfall 3] \hfill \\
I den tredje körningen blev kartan rätt och de moduler markerade med 1, 2, 3 och 4 i figur \ref{bana} blev avsökta (precis som de ska). Dock stannade roboten i den modul målet befinner sig i.

\pagebreak
\vspace{1em}
\item[Eventuella komplikationer] \hfill \\
Avsökningsalgoritmen tillåter inte att roboten på vägen tillbaka går genom målrutan.

\vspace{1em}

\item[Hur arbetet fortskrider] \hfill \\
\emph{Elena} och \emph{Lovisa} ska titta på hur det är implementerat att roboten ska ta sin väg tillbaka.


\end{description}

\end{flushleft}

\end{document}