\documentclass[11pt]{article}

\usepackage{extras} % Se extras.sty

\begin{document}

\begin{center}

{\Huge\bfseries Testprotokoll}
\vspace{4em}
\end{center}

\begin{flushleft}


\begin{figure}[htbp]
\centering
\noindent\resizebox{.4\linewidth}{!}{
	\documentclass[border=10px]{standalone}
\usepackage{tikz}
\usetikzlibrary{patterns}
\usetikzlibrary{shapes.arrows}
\usepackage{amssymb}
\begin{document}
	
\begin{tikzpicture}[scale=1]
					
\draw[thick] 	(0,5) -- (0,6);
\draw[thick] 	(2,0) -- (3,0);
\draw[thick] 	(5,5) -- (5,6);
\draw[thick]  (3,7) -- (4,7);		
\draw[thick,pattern=north west lines, pattern color=black]	 	(0,0) -- (2,0) -- (2,5) -- (0,5) -- (0,0);
\draw[thick,pattern=north west lines, pattern color=black]	 	(0,6) -- (0,7) -- (3,7) -- (3,6) -- (0,6);
\draw[thick,pattern=north west lines, pattern color=black]	 	(4,6) -- (4,7) -- (5,7) -- (5,6) -- (4,6);
\draw[thick,pattern=north west lines, pattern color=black]	 	(3,0) -- (5,0) -- (5,5) -- (3,5) -- (3,2) -- (4,2) -- (4,1) -- (3,1) -- (3,0);

\node at (4.5,5.5)[color=red] {1};
\node at (4.5,1.5)[color=red] {2};

\draw[thick,->] (0.2,5.5) -- (0.7,5.5);
	\end{tikzpicture}
	
\end{document}}
	\caption{Översikt av testbana\label{bana}}	
\end{figure}

\begin{description}
\item[Vad som har testats] \hfill \\
Förjustering och efterjustering av reglering testades samtidigt som måluppritning på karta i datormodulen i banan i figur \ref{bana}. 

\vspace{1em}
\item[Utfall] \hfill \\
Rroboten körde på direkt efter rotation. Målet markerades två gånger på datorns karta. Avsökningsalgoritmen verkade få fnatt. Var trodde roboten den var? 

\vspace{1em}
\item[Eventuella komplikationer] \hfill \\
\begin{itemize}
	\item På kartan var målet placerat både i ruta 1 och 2 markerade i figur \ref{bana}. På riktigt var den i ruta 1. Den placerades ut när det riktigta målet upptäcktes.
	\item Roboten körde inte riktigt som tanken är. Var befinner den sig?
	\item Efterrotationen fungerade inte då den körde på direkt efter rotation istället för att räta upp sig.
\end{itemize}

\vspace{1em}
\pagebreak
\item[Hur arbetet fortskrider] \hfill \\
\begin{itemize}
	\item Oklart... Börja med att titta vilka koordinater som datorn tar emot när rutorna ritas upp, dvs implementera så att det skrivs ut på väl valt ställe. \emph{Isak} fixar.
	\item Skicka position och riktining till datorn för att i kartan markera var den tror att den är någonstans. \emph{Elena} lägger in så att datan skickas, \emph{Isak} ser till att det tolkas på datorsidan.
	\item Jämförelse behöver ske med nya värden, inte de som justerats ned till noll under förregleringen. \emph{Olle} fixar.
\end{itemize}


\end{description}

\end{flushleft}

\end{document}
